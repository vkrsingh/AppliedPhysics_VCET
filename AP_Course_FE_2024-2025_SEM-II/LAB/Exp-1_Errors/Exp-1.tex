\documentclass[12pt]{article}
\usepackage{amsmath, amssymb, graphicx}
\usepackage{geometry}
\geometry{a4paper, margin=1in}
\usepackage{nopageno}
\begin{document}
	
\begin{center}
	\begin{minipage}{0.1\textwidth}
		\includegraphics[scale=0.1]{vcet-logo.jpeg}
	\end{minipage}
	\hfill 
	\begin{minipage}{0.85\textwidth}
		\textbf{\Large \textbf{\begin{center}  Vidyavardhini's College of Engineering and Technology, Vasai (West) \end{center} }}
	\end{minipage}
	
	\vspace{0.3cm}
	\textbf{\large First Year Engineering} \\
	\vspace{0.3cm}
	\textbf{\large Academic Year: 2024-2025} \\
	\vspace{0.5cm}
	\textbf{\Large \underline{Experiment No. 1: Statistical Treatment of Errors} }
\end{center}

	\section*{Objective}
	To assess understanding of statistical methods applied to a physics experiment involving sample mean, sample standard deviation, population mean, population standard deviation, and principles of least squares.
	
	\subsection*{Sample Mean and Sample Standard Deviation}
	\begin{enumerate}
		\item A pendulum is set up to measure the acceleration due to gravity. The period of 10 oscillations is measured 5 times for a pendulum length of 50 cm. The following times (in seconds) were recorded:
		\[ 12.1, 11.9, 12.0, 12.2, 12.1 \]
		\begin{enumerate}
			\item Calculate the sample mean of the period of one oscillation.
			\item Determine the sample standard deviation of the period.
		\end{enumerate}
	\end{enumerate}
	
	\subsection*{Population Mean and Standard Deviation}
	\begin{enumerate}
		\item Explain the difference between the sample mean and the population mean in the context of experimental physics.
		\item If the entire population of measurements were taken, and the mean period is found to be 12.05 seconds with a standard deviation of 0.15 seconds, how does this compare to your sample calculations?
	\end{enumerate}
	
	\subsection*{Principles of Least Squares}
	\begin{enumerate}
		\item The period of a pendulum is related to its length by the formula:
		\[ T^2 = \frac{4\pi^2}{g} L \]
		where \( T \) is the period, \( L \) is the length, and \( g \) is the acceleration due to gravity.
		The following data were recorded for three different lengths of the pendulum:
		\begin{center}
			\begin{tabular}{|c|c|}
				\hline
				Length (cm) & Period (s) \\
				\hline
				50 & 1.41 \\
				70 & 1.67 \\
				90 & 1.90 \\
				\hline
			\end{tabular}
		\end{center}
		\begin{enumerate}
			\item Calculate \( T^2 \) for each length.
			\item Plot \( T^2 \) against \( L \) and determine the slope of the best-fit line using the least squares method.
			\item Use the slope to calculate the acceleration due to gravity \( g \).
		\end{enumerate}
	\end{enumerate}
	
	\subsection*{Error Analysis and Discussion}
	\begin{enumerate}
		\item Discuss the possible sources of error in the pendulum experiment and their impact on the results.
		\item How do the sample standard deviation and the least squares method help in minimizing the effect of random errors?
	\end{enumerate}
	
\end{document}

