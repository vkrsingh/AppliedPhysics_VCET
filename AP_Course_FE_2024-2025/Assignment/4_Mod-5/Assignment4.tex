\documentclass[a4paper,12pt]{article}
\usepackage[margin=1in]{geometry}
\usepackage{amsmath,amsfonts}
\usepackage{multicol}
\usepackage{graphicx}
\usepackage{nopageno}
\begin{document}


\begin{center}
\begin{minipage}{0.1\textwidth}
\includegraphics[scale=0.1]{vcet-logo.jpeg}
\end{minipage}
 \hfill 
\begin{minipage}{0.85\textwidth}
 \textbf{\Large \textbf{\begin{center}  Vidyavardhini's College of Engineering and Technology, Vasai (West) \end{center} }}
\end{minipage}

\vspace{0.3cm}
\textbf{\large \underline{First Year Engineering}} \\
\vspace{0.3cm}
\textbf{\large Academic Year: 2024-2025} \\
\vspace{0.3cm}
\textbf{\large Assignment No: 04 } \\
\end{center}

\noindent
\textbf{Subject: BSC102/AP} \hfill { \textbf{Date: 18/11/2024} }\\
\textbf{Max Marks: 10} \hfill { \textbf{Duration: 1 Hr} }\\
\hrule

\vspace{0.2cm}
\noindent
\textbf{CO4:} Learners will be able to relate the foundation of quantum mechanics with the development of modern technology. \\
\vspace{0.3cm}
%\textbf{CO4:} Articulate the significance of Maxwell’s equations.

\begin{center}
\begin{tabular}{|p{0.5cm}|p{12cm}|c|p{0.5cm}|p{0.5cm}|}
\hline
\textbf{Q. No.} & \centering \textbf{Questions} & \textbf{Marks} & \textbf{CO} & \textbf{CL} \\
\hline

\begin{center} \textbf{1} \end{center}	& 
\centering Explain de-Broglie hypothesis of matter waves and deduce the expression for wavelength.  
& \textbf{3} & \textbf{4} &  \textbf{2} \\

\hline


\begin{center} \textbf{2} \end{center}	&  
\centering  Calculate de Broglie wavelength of an electron accelerated under the potential of 100 V0lts.  
& \textbf{2} & \textbf{4} & \textbf{3} \\

\hline

\begin{center} \textbf{3} \end{center}	& 
\centering State and explain Heisenberg’s Uncertainty Principle.  
 & \textbf{3} & \textbf{4} &  \textbf{2} \\
\hline

\begin{center} \textbf{4} \end{center}	&  
\centering  Calculate the uncertainty in the position of electron, if the speed of an electron is measured to be 4 x $10^5$ m/s to an accuracy of 0.002\%.
 & \textbf{2} & \textbf{4} & \textbf{3} \\

\hline

\end{tabular}
\end{center}

\hrule
\vspace{0.5cm}
\textbf{De Broglie Hypothesis of Matter Waves:} \textit{While we don't see waves for large objects like a ball due to their extremely small wavelengths, the concept is crucial for understanding technologies like electron microscopes. These microscopes use the wave nature of electrons (short wavelengths) to resolve tiny structures much smaller than visible light can, enabling advancements in materials science and biology.} \\

\textbf{Heisenberg’s Uncertainty Principle:} \textit{This principle is foundational to quantum mechanics and explains why electrons in atoms don't crash into the nucleus—they exist in probabilistic clouds. On a macroscopic level, it relates to precision limits in measurements. For example, in GPS technology, uncertainties in time measurements (using atomic clocks) can lead to slight deviations in location accuracy, which engineers minimize but can't completely eliminate.}


%\begin{center}
%\begin{tabular}{|p{0.5cm}|p{13cm}|c|p{0.5cm}|p{0.5cm}|}
%\hline
%
%\begin{center} \textbf{5} \end{center}	& 
%\begin{enumerate}
%\item[(a)] Find the curl of a vector field $\mathbf{E} = 4x \hat{i} + 2y \hat{j} + 3z \hat{k}$.
%
%\item[(b)] Calculate $\nabla \times \mathbf{A}$ at a point $(2, -2, 2)$ for $\mathbf{A} = x^2y \hat{i} - 3xyz^2 \hat{j} + 2xy \hat{k}$.
%\end{enumerate} & \textbf{4} & \textbf{4} & \textbf{3} \\

%\hline
%
%\end{tabular}
%\end{center}
%	
	
\end{document}
