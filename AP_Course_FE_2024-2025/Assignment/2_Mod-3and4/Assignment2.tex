\documentclass[a4paper,12pt]{article}
\usepackage[margin=1in]{geometry}
\usepackage{amsmath,amsfonts}
\usepackage{multicol}

\begin{document}
	
\begin{center}
\textbf{\Large \textbf{Vidyavardhini's College of Engineering and Technology, Vasai (West)}} \\
\vspace{0.3cm}
\textbf{\large \underline{First Year Engineering}} \\
\vspace{0.3cm}
\textbf{\large Academic Year: 2024-2025} \\
\vspace{0.3cm}
\textbf{\large Assignment No: 02 } \\
\end{center}

\noindent
\textbf{Subject: BSC102/AP} \hfill { \textbf{Date: 11/11/2024} }\\
\textbf{Max Marks: 30} \hfill { \textbf{Duration: 1 Hr} }\\
\hrule

\vspace{0.2cm}
\noindent
\textbf{CO3:} Determine the wavelength of light and refractive index of liquid using the interference phenomenon. \\
\vspace{0.3cm}
\textbf{CO4:} Articulate the significance of Maxwell’s equations.

\begin{center}
\begin{tabular}{|p{0.5cm}|p{13cm}|c|p{0.5cm}|p{0.5cm}|}
\hline
\textbf{Q. No.} & \centering \textbf{Questions} & \textbf{Marks} & \textbf{CO} & \textbf{CL} \\
\hline

\begin{center} \textbf{1} \end{center}	&  
\begin{enumerate}
\item[(a)] Find the thickness of the soap film which appears yellow (wavelength 5896 Å) in reflection when it is illuminated by white light at an angle of 45°. Given refractive index of the film is 1.33.

\item[(b)] A soap film $4 \times 10^{-5}$ cm thick is viewed at an angle of 35° to normal. Calculate the wavelength of light in the visible spectrum that will be absent from the reflected light ($\mu = 1.33$).
\end{enumerate}  & \textbf{4} & \textbf{3} &  \textbf{3} \\

\hline

\begin{center} \textbf{2} \end{center}	&  
\begin{enumerate}
\item[(a)]  In Newton’s Ring experiment, the wavelength of light incident is $5 \times 10^{-5}$ cm. If the diameter of the 10th dark ring is 0.5 cm, calculate the radius of curvature $R$. 

\item[(b)] In Newton's ring experiment, the diameter of the 15th ring was found to be 0.590 cm and that of the 5th ring was 0.336 cm. If the radius of the plano-convex lens is 100 cm, compute the wavelength of light used. 
\end{enumerate} & \textbf{4} & \textbf{3} & \textbf{3} \\

\hline

\begin{center} \textbf{3} \end{center}	& 
\begin{enumerate}
\item[(a)]  Find the gradient of a scalar field $\phi = x^2y + 4xy + xy^2z^2$. 

\item[(b)] Calculate $\nabla \cdot \mathbf{B}$ for $B = x^2 + y^2 + z^2$ at a point $(1, -2, 4)$.
\end{enumerate} & \textbf{4} & \textbf{4} & \textbf{3} \\
\hline

\begin{center} \textbf{4} \end{center}	&  
\begin{enumerate}
\item[(a)]  Find the divergence of a vector field $\mathbf{F} = 4x \hat{i} + 2y \hat{j} + 3z \hat{k}$. 

\item[(b)] Calculate $\nabla \cdot \mathbf{A}$ at a point $(1, -2, 2)$ for $\mathbf{A} = x^2y \hat{i} - 3xyz^2 \hat{j} + 2xy \hat{k}$. 
\end{enumerate} & \textbf{4} & \textbf{4} & \textbf{3} \\

\hline

\end{tabular}
\end{center}

\begin{center}
\begin{tabular}{|p{0.5cm}|p{13cm}|c|p{0.5cm}|p{0.5cm}|}
\hline

\begin{center} \textbf{5} \end{center}	& 
\begin{enumerate}
\item[(a)] Find the curl of a vector field $\mathbf{E} = 4x \hat{i} + 2y \hat{j} + 3z \hat{k}$.

\item[(b)] Calculate $\nabla \times \mathbf{A}$ at a point $(2, -2, 2)$ for $\mathbf{A} = x^2y \hat{i} - 3xyz^2 \hat{j} + 2xy \hat{k}$.
\end{enumerate} & \textbf{4} & \textbf{4} & \textbf{3} \\

\hline

\end{tabular}
\end{center}
	
	
\end{document}
